\documentclass[a4paper,12pt]{article}

\usepackage[utf8]{inputenc}
\usepackage[T1]{fontenc}
\usepackage[ngerman]{babel}
\usepackage{amsmath, amssymb}
\usepackage{graphicx}
\usepackage{geometry}
\usepackage{listings}
\usepackage{tikz}
\usetikzlibrary{arrows.meta}

\geometry{a4paper, margin=2.5cm}

\title{Aufgabenblatt 7\\Technische Informatik II}
\author{Tobias Petsch}
\date{}

\begin{document}

\maketitle

\section*{Aufgabe 7.1}

\subsection*{a)}

Die Parkplatz Simulation bricht ab, da durch eine Race Condition der \lstinline{sem->value} negativ wird und so ein Deadlock entsteht.
Wenn mehrere Threads feststellen, dass \lstinline{sem->value >= 0} ist dann werden sie den Wert verringern was zu einem negativen Wert führt.

\section*{Aufgabe 7.2}

\subsection*{a)}

Da jeder Philosoph erst sein linkes Stäbchen nimmt, kommt es zu Situationen bei denen das zweite Stäbchen vom Nachbarn genommen wird und das solange bis 
jeder Philosoph genau ein Stäbchen in der Hand hat, da er es nicht ablegen darf bis er gegessen hat, verhungert er.

\subsection*{b)}

\begin{center}
\begin{tikzpicture}[>=stealth, node distance=2cm, every node/.style={font=\small}]
  % Philosophen
  \node[circle, draw] (P1) at (90:3cm) {P1};
  \node[circle, draw] (P2) at (18:3cm) {P2};
  \node[circle, draw] (P3) at (-54:3cm) {P3};
  \node[circle, draw] (P4) at (-126:3cm) {P4};
  \node[circle, draw] (P5) at (162:3cm) {P5};

  % Stäbchen
  \node[rectangle, draw] (S1) at (90:1.5cm) {S1};
  \node[rectangle, draw] (S2) at (18:1.5cm) {S2};
  \node[rectangle, draw] (S3) at (-54:1.5cm) {S3};
  \node[rectangle, draw] (S4) at (-126:1.5cm) {S4};
  \node[rectangle, draw] (S5) at (162:1.5cm) {S5};

  % Besitz: durchgezogene Pfeile (Stäbchen -> Philosoph)
  \draw[->] (S1) -- (P1);
  \draw[->] (S2) -- (P2);
  \draw[->] (S3) -- (P3);
  \draw[->] (S4) -- (P4);
  \draw[->] (S5) -- (P5);

  % Anfragen: gestrichelte Pfeile (Philosoph -> Stäbchen)
  \draw[->, dashed] (P1) -- (S2);
  \draw[->, dashed] (P2) -- (S3);
  \draw[->, dashed] (P3) -- (S4);
  \draw[->, dashed] (P4) -- (S5);
  \draw[->, dashed] (P5) -- (S1);

\end{tikzpicture}
\end{center}

\subsection*{c)}

\begin{verbatim}
    while true do
        wait until keiner der Nachbarn einen Hut trägt
        setze lustigen Hut auf

        warte, bis beide Stäbchen verfügbar sind (keine Konflikte mit Nachbarn)
    
        nimm das linke Stäbchen
        nimm das rechte Stäbchen
    
        esse Reis, bis du satt bist

        lege beide Stäbchen zurück auf den Tisch
        setze den Hut ab
        denke nach, bis du wieder hungrig bist

\end{verbatim}

\subsection*{d)}

Wenn zwei Philosophen abwechselnd immer essen, ohne dabei den mittleren Philosophen zu berücksichtigen, dann wird dieser verhungern da er nie zum Zug kommt.

\subsection*{e)}

\begin{center}
\begin{tikzpicture}[>=stealth, node distance=2cm, every node/.style={font=\small}]
  % Philosophen
  \node[circle, draw] (P1) at (90:3cm) {P1};
  \node[circle, draw] (P2) at (18:3cm) {P2};
  \node[circle, draw] (P3) at (-54:3cm) {P3}; % Der Rechtshänder
  \node[circle, draw] (P4) at (-126:3cm) {P4};
  \node[circle, draw] (P5) at (162:3cm) {P5};

  % Stäbchen
  \node[rectangle, draw] (S1) at (90:1.5cm) {S1};
  \node[rectangle, draw] (S2) at (18:1.5cm) {S2};
  \node[rectangle, draw] (S3) at (-54:1.5cm) {S3};
  \node[rectangle, draw] (S4) at (-126:1.5cm) {S4};
  \node[rectangle, draw] (S5) at (162:1.5cm) {S5};

  % Besitz: durchgezogene Pfeile (Stäbchen -> Philosoph)
  \draw[->] (S1) -- (P1);
  \draw[->] (S2) -- (P2);
  \draw[->] (S4) -- (P4);
  \draw[->] (S5) -- (P5);

  % P3 nimmt zuerst rechtes Stäbchen (S4)
  \draw[->] (S4) -- (P3);

  % Anfragen: gestrichelte Pfeile (Philosoph -> Stäbchen)
  \draw[->, dashed] (P1) -- (S2);
  \draw[->, dashed] (P2) -- (S3);

  % P3 fordert linkes (S3) — da er rechts zuerst genommen hat
  \draw[->, dashed] (P3) -- (S3);

  \draw[->, dashed] (P4) -- (S5);
  \draw[->, dashed] (P5) -- (S1);

\end{tikzpicture}
\end{center}

Es kann kein Deadlock mehr entstehen, da kein zyklischer Deadlock entstehen kann.


\section*{Aufgabe 7.3}

Die \texttt{<stdatomic>} Bibliothek erlaubt es atomare Operationen auf Variablen auszuführen, die frei von Race Conditions sind.
Sie erlaubt das sichere Schreiben und Lesen von Variablen mithilfe von \texttt{atomic\_int} oder ähnlichen. Sie ist 
hilfreich für multithreaded Programme oder auch Zähler.

\section*{Aufgabe 7.4}

\subsection*{a)}

\begin{itemize}
    \item Mutual exclusion: Ressourcen können immer nur von einem Prozess gleichzeitig genutzt werden.
    \item Hold and Wait: Ein Prozess hält bereits mindestens eine Ressource und wartet zusätzlich auf weitere, die von anderen Prozessen belegt sind.
    \item No Preemption: Ressourcen können einem Prozess nicht zwangsweise entzogen werden. Sie müssen freiwillig freigegeben werden.
    \item Circular Wait: Es existiert eine zyklische Wartekette, in der jeder Prozess auf eine Ressource wartet, die von einem anderen Prozess in der Kette gehalten werden.
\end{itemize}



\subsection*{b)}

\begin{itemize}
  \item \textbf{Mutual Exclusion vermeiden:}
  \begin{itemize}
    \item Betriebsmittel werden, wenn möglich, so gestaltet, dass sie gleichzeitig von mehreren Prozessen genutzt werden können (z.\,B. durch Read-Only-Zugriffe).
    \item Dadurch ist keine exklusive Nutzung erforderlich.
    \item \textit{Die Bedingung ist nie erfüllt, weil keine Ressource exklusiv vergeben wird.}
  \end{itemize}

  \item \textbf{Hold and Wait vermeiden:}
  \begin{itemize}
    \item Prozesse müssen vor Ausführung alle benötigten Ressourcen auf einmal anfordern.
    \item Falls nicht alle verfügbar sind, wird der Prozess blockiert und bekommt keine.
    \item \textit{Die Bedingung ist nie erfüllt, da Prozesse keine Ressourcen halten, während sie auf andere warten.}
  \end{itemize}

  \item \textbf{No Preemption vermeiden:}
  \begin{itemize}
    \item Ressourcen können einem Prozess entzogen werden, wenn dieser auf weitere wartet.
    \item Das Betriebssystem kann die Ressource zwangsweise freigeben und später neu zuteilen.
    \item \textit{Die Bedingung ist nie erfüllt, da Prozesse Ressourcen nicht dauerhaft behalten dürfen.}
  \end{itemize}

  \item \textbf{Circular Wait vermeiden:}
  \begin{itemize}
    \item Eine feste Reihenfolge (Total Order) für die Anforderung von Ressourcen wird definiert.
    \item Prozesse dürfen nur in aufsteigender Reihenfolge nach Ressourcen fragen.
    \item \textit{Die Bedingung ist nie erfüllt, weil dadurch keine zyklischen Wartekanten entstehen können.}
  \end{itemize}
\end{itemize}






\end{document}