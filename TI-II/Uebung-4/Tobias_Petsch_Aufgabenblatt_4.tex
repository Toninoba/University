\documentclass[a4paper,12pt]{article}

% Packages
\usepackage[utf8]{inputenc}
\usepackage[T1]{fontenc}
\usepackage[ngerman]{babel}
\usepackage{amsmath, amssymb}
\usepackage{graphicx}
\usepackage{geometry}
\usepackage{hyperref}
\usepackage{float}

% Page layout
\geometry{a4paper, left=25mm, right=25mm, top=25mm, bottom=25mm}

% Title
\title{Aufgabenblatt 4}
\author{Tobias Petsch}
\date{}

\begin{document}

\maketitle

\section*{Aufgabe 1}
fork() erzeugt ein Kindsprozess, welcher einer PID bekommt und von dieser von seinem Elternprozess unterscheidet.
Er erstellt also einen neuen Thread.

\section*{Aufgabe 2}

\subsection*{a)}

Nur der Textbereich und Data liegen tatsächlich auf dem Festplattenspeicher, der Rest wird nach Start im Arbeitsspeicher abgelegt.

\subsection*{b)}

\begin{table}[H]
    \centering
    \begin{tabular}{|c|c|}
    \hline
    text & 298 \\
    data   & 4   \\
    bss   & 8   \\
    dec & 310\\
    hex & 136 \\
    \hline
    \end{tabular}
\end{table}


\end{document}