\documentclass[a4paper,12pt]{article}

\usepackage[utf8]{inputenc}
\usepackage[T1]{fontenc}
\usepackage[ngerman]{babel}
\usepackage{amsmath, amssymb}
\usepackage{graphicx}
\usepackage{geometry}
\usepackage{tikzinclude}
\geometry{a4paper, margin=2.5cm}

\title{Aufgabenblatt 6\\Technische Informatik II}
\author{Tobias Petsch}
\date{}

\begin{document}

\maketitle



\section*{Aufgabe 6.1}

\begin{itemize}
  \item \textbf{Grau}: Wartezeit
  \item \textbf{Grün}: Bearbeitungszeit
\end{itemize}

\subsection*{a)}



\begin{center}
    \begin{tikzpicture}

    \draw[->] (0,0) -- (16,0) node[right] {Zeit};
    \draw[->] (0,0) -- (0,6) node[above] {Prozesse};

    % Prozess 1
  \filldraw[fill=green!30, draw=black] (0,5) rectangle (3,5.8);
  \node at (1.5,5.4) {A};

  % Prozess 2
  \filldraw[fill=gray!30, draw=black] (2,4) rectangle (3,4.8);
  \filldraw[fill=green!30, draw=black] (3,4) rectangle (5,4.8);
  \node at (4,4.4) {B};

  % Prozess 3
  \filldraw[fill=gray!30, draw=black] (2,3) rectangle (5,3.8);
  \filldraw[fill=green!30, draw=black] (5,3) rectangle (6,3.8);
  \node at (5.5,3.4) {C};

  \filldraw[fill=gray!30, draw=black] (5,2) rectangle (6,2.8);
  \filldraw[fill=green!30, draw=black] (6,2) rectangle (14,2.8);
  \node at (10,2.4) {D};

  \filldraw[fill=gray!30, draw=black] (8,1) rectangle (14,1.8);
  \filldraw[fill=green!30, draw=black] (14,1) rectangle (15,1.8);
  \node at (14.5,1.4) {E};

  % Zeitmarkierungen
  \foreach \x in {0,1,...,15} {
    \draw (\x,0.1) -- (\x,-0.1) node[below] {\x};
  }

\end{tikzpicture}
\end{center}



\subsection*{b)}

\begin{center}
    \begin{tikzpicture}

    \draw[->] (0,0) -- (16,0) node[right] {Zeit};
    \draw[->] (0,0) -- (0,6) node[above] {Prozesse};

  % Prozess A
  \filldraw[fill=green!30, draw=black] (0,5) rectangle (3,5.8);
  \node at (1.5,5.4) {A};

  % Prozess B
  \filldraw[fill=gray!30, draw=black] (3,4) rectangle (4,4.8);
  \filldraw[fill=green!30, draw=black] (4,4) rectangle (6,4.8);
  \node at (5,4.4) {B};

  % Prozess C
  \filldraw[fill=green!30, draw=black] (3,3) rectangle (4,3.8);
  \node at (3.5,3.4) {C};

  % Prozess D
  
  \filldraw[fill=green!30, draw=black] (6,2) rectangle (14,2.8);
  \node at (10,2.4) {D};

  % Prozess E
  \filldraw[fill=gray!30, draw=black] (9,1) rectangle (14,1.8);
  \filldraw[fill=green!30, draw=black] (14,1) rectangle (15,1.8);
  \node at (14.5,1.4) {E};

  % Zeitmarkierungen
  \foreach \x in {0,1,...,15} {
    \draw (\x,0.1) -- (\x,-0.1) node[below] {\x};
  }

\end{tikzpicture}
\end{center}

\subsection*{c)}

\begin{center}
  \begin{tikzpicture}

  \draw[->] (0,0) -- (16,0) node[right] {Zeit};
  \draw[->] (0,0) -- (0,6) node[above] {Prozesse};

  % Prozess A
  \filldraw[fill=green!30, draw=black] (0,5) rectangle (3,5.8);
  \node at (1.5,5.4) {A};

  % Prozess B
  \filldraw[fill=gray!30, draw=black] (3,4) rectangle (4,4.8);
  \filldraw[fill=green!30, draw=black] (4,4) rectangle (6,4.8);
  \node at (5,4.4) {B};

  % Prozess C
  \filldraw[fill=green!30, draw=black] (3,3) rectangle (4,3.8);
  \node at (3.5,3.4) {C};

  % Prozess D
  \filldraw[fill=green!30, draw=black] (6,2) rectangle (8,2.8);
  \filldraw[fill=gray!30, draw=black] (8,2) rectangle (9,2.8);
  \filldraw[fill=gray!30, draw=black] (5,2) rectangle (6,2.8);
  \filldraw[fill=green!30, draw=black] (9,2) rectangle (15,2.8);
  \node at (12.5,2.4) {D};

  % Prozess E
  \filldraw[fill=green!30, draw=black] (8,1) rectangle (9,1.8);
  \node at (8.5,1.4) {E};

  % Zeitmarkierungen
  \foreach \x in {0,1,...,15} {
  \draw (\x,0.1) -- (\x,-0.1) node[below] {\x};
  }

\end{tikzpicture}
\end{center}

\subsection*{d)}

\begin{center}
    \begin{tikzpicture}

    \draw[->] (0,0) -- (16,0) node[right] {Zeit};
    \draw[->] (0,0) -- (0,6) node[above] {Prozesse};

  % Prozess A
  \filldraw[fill=green!30, draw=black] (0,5) rectangle (2,5.8);
  \filldraw[fill=green!30, draw=black] (4,5) rectangle (5,5.8);
  \node at (1,5.4) {A};
  \node at (4.5,5.4) {A};

  % Prozess B
  \filldraw[fill=green!30, draw=black] (2,4) rectangle (3,4.8);
  \node at (2.5,4.4) {B};
  \filldraw[fill=green!30, draw=black] (5,4) rectangle (6,4.8);
  \node at (5.5,4.4) {B};

  % Prozess C
  \filldraw[fill=green!30, draw=black] (3,3) rectangle (4,3.8);
  \node at (3.5,3.4) {C};

  % Prozess D
  \filldraw[fill=green!30, draw=black] (6,2) rectangle (8,2.8);
  \filldraw[fill=green!30, draw=black] (9,2) rectangle (15,2.8);
  \node at (12,2.4) {D};
  \node at (7,2.4) {D};

  % Prozess E
  \filldraw[fill=green!30, draw=black] (8,1) rectangle (9,1.8);
  \node at (8.5,1.4) {E};

  % Zeitmarkierungen
  \foreach \x in {0,1,...,15} {
    \draw (\x,0.1) -- (\x,-0.1) node[below] {\x};
  }

\end{tikzpicture}
\end{center}

\subsection*{e)}

\begin{center}
\renewcommand{\arraystretch}{1.2}
\begin{tabular}{|c|c|c|c|}
\multicolumn{4}{c}{\textbf{FCFS}} \\
\hline
 & Finish Time & Tr & Tr/Ts \\
\hline
A & 3 & 3 & 1\\
\hline
B & 5 & 3 & 1.66\\
\hline
C & 6 & 4 & 1.5\\
\hline
D & 14 & 9 & 1.55\\
\hline
E & 15 & 7 & 2.143\\
\hline
MW & & 5.2 & 1.5706  \\
\hline
\end{tabular}
\end{center}

\begin{center}
\renewcommand{\arraystretch}{1.2}
\begin{tabular}{|c|c|c|c|}
\multicolumn{4}{c}{\textbf{SPN}} \\
\hline
 & Finish Time & Tr & Tr/Ts \\
\hline
A & 3 & 3 & 1\\
\hline
B & 6 & 4 & 1.5\\
\hline
C & 4 & 2 & 2\\
\hline
D & 14 & 9 &  1.55\\
\hline
E & 15 & 7 & 2.143\\
\hline
MW & & 5 & 1.6386  \\
\hline
\end{tabular}
\end{center}


\begin{center}
\renewcommand{\arraystretch}{1.2}
\begin{tabular}{|c|c|c|c|}
\multicolumn{4}{c}{\textbf{SRT}} \\
\hline
 & Finish Time & Tr & Tr/Ts \\
\hline
A & 3 & 3 & 1\\
\hline
B & 6 & 4 & 1.5\\
\hline
C & 4 & 2 & 2\\
\hline
D & 15 & 10 &  1.5\\
\hline
E & 9 & 1 & 9\\
\hline
MW & & 4 & 3 \\
\hline
\end{tabular}
\end{center}


\begin{center}
\renewcommand{\arraystretch}{1.2}
\begin{tabular}{|c|c|c|c|}
\multicolumn{4}{c}{\textbf{Round Robin}} \\
\hline
 & Finish Time & Tr & Tr/Ts \\
\hline
A & 5 & 5 & 1\\
\hline
B & 6 & 4 & 1.5\\
\hline
C & 4 & 2 & 2\\
\hline
D & 15 & 10 &  1.5\\
\hline
E & 9 & 1 & 9\\
\hline
MW & & 4.4 & 3\\
\hline
\end{tabular}
\end{center}


\section*{6.2}

\subsection*{a)}

\begin{center}
    \begin{tikzpicture}

    \draw[->] (0,0) -- (16,0) node[right] {Zeit};
    \draw[->] (0,0) -- (0,6) node[above] {Prozesse};

    % Prozess 1
  \filldraw[fill=green!30, draw=black] (0,5) rectangle (3,5.8);
  \node at (1.5,5.4) {A};

  % Prozess 2
  \filldraw[fill=gray!30, draw=black] (2,4) rectangle (3,4.8);
  \filldraw[fill=red!30, draw=black] (3,4) rectangle (5,4.8);
  \node at (4,4.4) {B};

  % Prozess 3
  \filldraw[fill=gray!30, draw=black] (2,3) rectangle (5,3.8);
  \filldraw[fill=red!30, draw=black] (5,3) rectangle (6,3.8);
  \node at (5.5,3.4) {C};

  \filldraw[fill=gray!30, draw=black] (5,2) rectangle (6,2.8);
  \filldraw[fill=green!30, draw=black] (6,2) rectangle (14,2.8);
  \node at (10,2.4) {D};

  \filldraw[fill=gray!30, draw=black] (8,1) rectangle (14,1.8);
  \filldraw[fill=red!30, draw=black] (14,1) rectangle (15,1.8);
  \node at (14.5,1.4) {E};

  % Zeitmarkierungen
  \foreach \x in {0,1,...,15} {
    \draw (\x,0.1) -- (\x,-0.1) node[below] {\x};
  }

\end{tikzpicture}
\end{center}

Alle mit rot markiereten Prozesse werden nicht korrekt beendet.


\subsection*{b)}

\begin{center}
    \begin{tikzpicture}

    \draw[->] (0,0) -- (16,0) node[right] {Zeit};
    \draw[->] (0,0) -- (0,6) node[above] {Prozesse};

    % Prozess 1
  \filldraw[fill=green!30, draw=black] (0,5) rectangle (3,5.8);
  \node at (1.5,5.4) {A};

  % Prozess 2
  \filldraw[fill=gray!30, draw=black] (2,4) rectangle (3,4.8);
  \filldraw[fill=red!30, draw=black] (3,4) rectangle (5,4.8);
  \node at (4,4.4) {B};

  % Prozess 3
  \filldraw[fill=gray!30, draw=black] (2,3) rectangle (5,3.8);
  \filldraw[fill=red!30, draw=black] (5,3) rectangle (6,3.8);
  \node at (5.5,3.4) {C};

  \filldraw[fill=gray!30, draw=black] (5,2) rectangle (6,2.8);
  \filldraw[fill=green!30, draw=black] (6,2) rectangle (14,2.8);
  \node at (10,2.4) {D};

  \filldraw[fill=gray!30, draw=black] (8,1) rectangle (14,1.8);
  \filldraw[fill=red!30, draw=black] (14,1) rectangle (15,1.8);
  \node at (14.5,1.4) {E};

  % Zeitmarkierungen
  \foreach \x in {0,1,...,15} {
    \draw (\x,0.1) -- (\x,-0.1) node[below] {\x};
  }

\end{tikzpicture}
\end{center}


\subsection*{c)}

\begin{center}
    \begin{tikzpicture}

    \draw[->] (0,0) -- (16,0) node[right] {Zeit};
    \draw[->] (0,0) -- (0,6) node[above] {Prozesse};

    % Prozess 1
  \filldraw[fill=green!30, draw=black] (0,5) rectangle (2,5.8);
  \filldraw[fill=green!30, draw=black] (5,5) rectangle (6,5.8);
  \node at (1.5,5.4) {A};

  % Prozess 2
  \filldraw[fill=green!30, draw=black] (2,4) rectangle (3,4.8);
  \filldraw[fill=red!30, draw=black] (3,4) rectangle (4,4.8);
  \node at (3.5,4.4) {B};

  % Prozess 3
  \filldraw[fill=gray!30, draw=black] (2,3) rectangle (4,3.8);
  \filldraw[fill=red!30, draw=black] (4,3) rectangle (5,3.8);
  \node at (4.5,3.4) {C};

  \filldraw[fill=gray!30, draw=black] (5,2) rectangle (6,2.8);
  \filldraw[fill=green!30, draw=black] (6,2) rectangle (8,2.8);
  \filldraw[fill=green!30, draw=black] (9,2) rectangle (15,2.8);
  \node at (12,2.4) {D};

  \filldraw[fill=green!30, draw=black] (8,1) rectangle (9,1.8);
  \node at (8.5,1.4) {E};

  % Zeitmarkierungen
  \foreach \x in {0,1,...,15} {
    \draw (\x,0.1) -- (\x,-0.1) node[below] {\x};
  }

\end{tikzpicture}
\end{center}


\section*{6.3}

\subsection*{a)}

\texttt{SCHED\_FIFO} = FCFS \\

\texttt{SCHED\_RR} = Round Robin \\

\texttt{SCHED\_OTHER} = Normal (Completly fair Scheduler) \\

\subsection*{b)}

\texttt{SCHED\_FIFO} = Rate Monotonic Scheduling \\

\texttt{SCHED\_RR} = (i) \\

\texttt{SCHED\_OTHER} = (iii) \\


\subsection*{c)}

zu FIFO:
  Feste Prioritaeten, nicht zeitscheibengesteuert, kein preemtion unter prozessen


zu RR:
  Feste Prioritaeten, preemtion
  
zu DEADLINE:
  basierend auf Deadline


\section*{6.4}

\subsection*{a)}

In der zweiten Simulation verhungern die Prozesse sehr wahrscheinlich weil 5 Prozesse gleichzeitig
laufen, was in der ersten Simulation nicht der Fall ist. Zudem haben die Prozesse in der zweiten Simulation 
ein Atomic Fetch add. 

\subsection*{b)}

Die Prozesse verhungern nur zum Teil, aber sie werden beendet. Da hier nicht auf die Beendigung eines Prozess gewartet wird, damit
andere Prozesse drankommen, werden alle Prozesse am Ende korrekt beendet.

\end{document}


