\documentclass[a4paper,12pt]{article}

% Packages
\usepackage[utf8]{inputenc}
\usepackage[ngerman]{babel}
\usepackage{amsmath, amssymb}
\usepackage{graphicx}
\usepackage{geometry}
\usepackage{hyperref}

% Page layout
\geometry{a4paper, margin=1in}

% Title
\title{Aufgabenblatt 3}
\author{Tobias Petsch}
\date{}

\begin{document}

\maketitle

\section*{Aufgabe 3.1}

\begin{table}[h!]
\centering
\begin{tabular}{|c|c|c|c|c|}
\hline
    & 0 & 1 & 2 & 3 \\ \hline
0x3c  &          &          &          &          \\ \hline
0x38  &          &          &          &          \\ \hline
0x34  &          &          &          &          \\ \hline
0x30  &          &          &          &          \\ \hline
0x2c  &          &          &          &          \\ \hline
0x28  &          &          &          &          \\ \hline
0x24  &          &          &          &          \\ \hline
0x20  &     4     &    7      &    8      &          \\ \hline
0x1c  &          &          &          &          \\ \hline
0x18  &          &          &          &          \\ \hline
0x14  &     3     &     2     &    5      &          \\ \hline
0x10  &          &     1     &          &          \\ \hline
0x0c  &          &          &          &          \\ \hline
0x08  &          &          &          &          \\ \hline
0x04  &          &          &          &          \\ \hline
0x00  &          &          &          &          \\ \hline
\end{tabular}

\end{table}

\section*{3.2}
\subsection*{a)}

Es wurde size\_t genutzt, da size\_t einen unsigned long long beschreibt und somit nicht negativ initialisiert werden kann. Außerdem
können so 64 bit vollständig addressiert werden. 


\section*{3.4}
\subsection*{a)}

Ich erwarte das eine Ausgabe fuer p4 mit 8 und fuer p8 mit 16 entsteht, da f() zahlen doppeln soll.

\subsection*{b)}

Ich könnte mir vorstellen das die initialisiert Variable b in f() gelöscht wird, die ihr Scope zuende geht, sie auf dem Stackspeicher liegt und somit
gelöscht werden kann. Dadurch entsteht ein dangling Pointer welcher nun irgendwo im Speicher hinzeigt.

\section*{3.6}

\subsection*{a)}
\texttt{cat legalcode.txt | tr -d '()"?.!:;,+\&.' | \\
tr '`-' ' ' | tr 'A-Z' 'a-z' > legalcode\_edited.txt} \\
Ausgabe durch cat legalcode\_edited.txt

\subsection*{b)}
wc legalcode\_edited.txt \\
319 2981 18996 legalcode\_edited.txt \\
319 Zeilen \\
2981 Woerter \\
18996 Zeichen \\

grep -vi 'the' legalcode\_edited.txt | wc -l \\
127 Zeilen

\subsection*{c)}
Alle Woerter in einzelne Zeilen packen

\texttt{tr ' ' '\textbackslash{}n' < legalcode\_edited.txt | grep -v '\textasciicircum{}\$' > legalcode\_edited\_seperated.txt}

\texttt{sort legalcode\_edited\_seperated.txt | uniq | wc -l} 

fuehrt zu einem Ergebnis von 666 unterschiedlichen Woertern

\texttt{sort legalcode\_edited\_seperated.txt | uniq -c | sort -nr}
244 the \\
119 or \\
116 of\\
102 to \\
83 work \\
68 a \\
67 license \\
61 in \\
58 and \\
55 this \\

\subsection*{d)}
\texttt{\detokenize{awk '{ sum += length($0); count++ } END { if (count > 0) print sum / count }' ausgabe_ohne_leerzeilen.txt}} \\
fuehrt zu 5.03958

\subsection*{e)}
\texttt{cat legalcode\_edited.txt | tr -d '[:space:][:punct:]' | \\
grep -o . | sort | uniq -c | sort -nr} \\
1738 e\\
1431 t\\
1318 o\\



\end{document}