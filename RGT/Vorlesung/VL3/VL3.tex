\documentclass[11pt]{article}

% Basic packages
\usepackage[utf8]{inputenc}
\usepackage[german]{babel}
\usepackage{amsmath}
\usepackage{amsfonts}
\usepackage{amssymb}
\usepackage{graphicx}
\usepackage{hyperref}

\setlength{\parindent}{0pt}

% Document information
\title{Regelungstechnik - Vorlesung 3}
\author{Tobi}
\date{\today}

\begin{document}

\maketitle

\section{LTI Systeme}

\begin{itemize}
    \item Übertragunsfunktion mithilfe von Laplace Transformation, musst du dir anschauen
    \item Was genau ist eine Zustandstransformation
\end{itemize}

\section{Lösung von LTI Systemen}

\begin{itemize}
    \item 
\end{itemize}

\end{document}\section{Einleitung}
% Hier beginnt der Inhalt Ihrer Vorlesung