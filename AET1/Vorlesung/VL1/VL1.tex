\documentclass[a4paper,12pt]{article}

% Packages
\usepackage[utf8]{inputenc}
\usepackage[T1]{fontenc}
\usepackage[ngerman]{babel}
\usepackage{amsmath}
\usepackage{amssymb}
\usepackage{graphicx}
\usepackage{hyperref}
\usepackage{geometry}


% Page layout
\geometry{a4paper, margin=1in}

\setlength{\parindent}{0pt}

% Title
\title{Vorlesung 1}
\author{Tobi}
\date{\today}

\begin{document}

\maketitle

\section{Organisatorisches}

Literaturvorschläge im Elearning

\subsection*{Seminararbeit}

scheinbar muss man jetzt die Seminararbeit machen.
Prüfungsvorleistung? Sollte in Studienordnung stehen.

\section{GRundbegriffe}

\subsection*{Ladung und Strom}

\begin{itemize}
    \item Strom ist die Bewegung von Ladungsträgern.
    \item Elektronen in Metallen und Ionen in Flüssigkeiten führen zur Bewegung der Ladungsträger.
    \item Stromfluss ist gleichzusetzen mit Volumen pro Zeit: 
    \[
    i = \frac{dQ}{dt}
    \]
\end{itemize}

\begin{itemize}
    \item Positive Ladungsträger bewegen sich von links nach rechts, der Pfeil zeigt die Stromrichtung an.
    \item Normalerweise zeigt der Pfeil in Richtung des Stromflusses, wenn $i$ positiv ist.
    \item Der Stromfluss kann negativ sein, obwohl der Stromfluss gleich bleibt.
    \item Stromfluss ist nicht gleich Stromrichtung.
    \item Wenn sich Ladungsträger von links nach rechts bewegen, ist links mehr positive Ladung als rechts.
    \item Es gibt Gleichstrom mit überlagertem Wechselanteil, d.h. die Summe der Fläche ist nicht gleich null.
    \item Großbuchstaben werden für konstante Größen verwendet.
    \item Amplitude bei Wechselstrom Großbuchstabe
\end{itemize}

\subsection*{Potential und Spannung}

\begin{itemize}
    \item Analogie Druckunterschied Wasserrohr, Potential auf einer Seite größer(negativer) als auf anderer Seite
    \item Druck = Spannung $u$
    \item positive Spannung von plus zu minus fließender Strom $\rightarrow$ $i = 0$
    \item Gegenteil umgekehrt
    \item Spannung = Potentialdifferenz
\end{itemize}

\subsection*{Widerstand}

\begin{itemize}
    \item Widerstand begrenzt Fluß 
    \item Ohmsches Gesetz $R = \frac{u}{i}$
    \item Widerstand nimmt mit steigender Länge zu und mit steigender Querschnittsfläche $A$ am
    \item $R = \rho \frac{l}{A}$
    \item $\rho$ = materialspezifischer Widerstand, Einheit $\Omega m$
    \item ändert sich mit Temperatur
    \item Kehrwert: Leitwert $G = \frac{1}{R}$
    \item Einheit $S$ Siemens
    \item gleiche Gleichung wie bei Widerstand mit Länge und Fläche
    \item mit $k$ als Temperaturabhängige Materialkonstante
\end{itemize}

\subsection*{Leistung und Energie}

\begin{itemize}
    \item $F = (p1 - p2) * A$
    \item innerhalb Zeitintervals wird Volumen verschoben
\end{itemize}

\subsection*{Quellen}

\begin{itemize}
    \item Spannungsquelle gibt konstante Spannung ab, Stromquelle\dots
    \item Erzeuger/Verbraucher Bezugspfeil System bestimmt ob etwas Leistung aufnimmt oder ob die Quelle Leistung aufnimmt
    \item kann auch beides in einem Bild sein, Bsp. Widerstand verbracuherpfeilsystm, Quelle Erzeuger\dots
\end{itemize}








\end{document}