\documentclass[a4paper,12pt]{article}

% Packages
\usepackage[utf8]{inputenc}
\usepackage[T1]{fontenc}
\usepackage[ngerman]{babel}
\usepackage{amsmath, amssymb}
\usepackage{graphicx}
\usepackage{hyperref}
\usepackage{geometry}

% Page layout
\geometry{a4paper, margin=1in}
\setlength{\parindent}{0pt}
% Title
\title{VL2 - Vorlesung}
\author{Tobi}
\date{\today}

\begin{document}

\maketitle

\section*{Stromkreise}

\subsection*{Grundregeln}

\begin{itemize}
    \item Bing Bong Was sind Knoten und Maschen das kennst du
    \item Knotenregel:
    \begin{itemize}
        \item $i_1 = i_2 + i_3$ 
        \item Die Summe der Teilströme gleich des Gesamtstroms
        \item Aber teilt sich nie gleich auf, musst du selbst berechnen du dummmer
    \end{itemize}
    \item Maschenregel:
    \begin{itemize}
        \item Druckdifferenz auf zwei Zweigen ist gleich 0
        \item Pfeilrichtung komisch aber macht schon irgendwo Sinn
        \item Die Summe aller Spannungen beo einem vollständigen Umlauf in einer Masche ist Null
        \item Pfeilrichtung beachten da Vorzeichen sich ändern, Umlauf ungleich Spannungsrichtung
    \end{itemize}
    
\end{itemize}

\subsection*{Paralellschaltung}

\begin{itemize}
    \item Spannung bleibt gleich, Stärke teilt sich auf 
    \item streicht sich viel raus bei Gesamtstrom und Gesamtwiderstand
    \item komischer Bruch wird auf einen Nenner gemacht und dann Doppelbruch aufgelöst. Merke dir Endprodukt für 2 Widerstände Formel Dings
    \item lol
    \item Gesamtwiderstand immer kleiner als Teilwiderstände einzeln, merken du dumbo
    \item anteiliger Stromfluß $\frac{i_1}{i_0}$ wie viel Anteil hat $i_1$ an $i_0$
    \item wird auch Stromteiler genannt
\end{itemize}

\subsection*{Reihenschaltung}

\begin{itemize}
    \item Strom bleibt gleich, Spannung teilt sich auf bzw. nimmt ab 
    \item ja bro war abgelenkt sorry
    \item Jetzt labert er was von Quellen und Leistungsanpassung
    \item kein plan brudi
    \item und wieder nicht aufgepasst schere
\end{itemize}

\end{document}