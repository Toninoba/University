\documentclass[a4paper,12pt]{article}

% Packages
\usepackage[utf8]{inputenc}
\usepackage[T1]{fontenc}
\usepackage[ngerman]{babel}
\usepackage{amsmath, amssymb}
\usepackage{graphicx}
\usepackage{hyperref}
\usepackage{geometry}

% Page layout
\geometry{a4paper, margin=1in}
\setlength{\parindent}{0pt}
% Title
\title{VL4 - Vorlesung}
\author{Tobi}
\date{\today}

\begin{document}

\maketitle
\section{Zeiger}



\begin{itemize}
    \item bei grundlegender Darstellung kann sinus weggelassen werden und omega, wichtig ist effektivwert U-Dach
    \item und phasenlage wichtig 
    \item Zeiger stellen beide wichtige Größen grafisch dar 
    \item Länge des Pfeils = Z = effektivwert
    \item mit Phasenwinkel alpha gedreht 
    \item j ist imaginäre einheit 
    \item Multiplikation mit j ist 90 grad drehung entgegen uhrzeiger 
    \item Rho ist Zeigerlänge = schiefes p 
    \item 
\end{itemize}



\end{document}