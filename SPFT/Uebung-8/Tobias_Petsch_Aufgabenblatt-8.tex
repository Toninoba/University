\documentclass[a4paper,12pt]{article}

\usepackage[utf8]{inputenc}
\usepackage[T1]{fontenc}
\usepackage[ngerman]{babel}
\usepackage{amsmath, amssymb}
\usepackage{geometry}
\geometry{a4paper, left=25mm, right=25mm, top=25mm, bottom=25mm}

\title{Aufgabenblatt 8\\\large SPFT}
\author{Tobias Petsch}
\date{}

\begin{document}

\maketitle

\section*{Aufgabe 1}

\begin{itemize}
    \item \textbf{Vorteile:}
    \begin{itemize}
        \item Programm kann direkt im Quelltext spezifiziert werden
        \item Automatische Methoden zum Überprüfen des Codes können ausgeführt werden
        \item einige Methoden funktionieren gut auf großen Programmen
        \item praktische Implementierung von theoretischen Methoden zur Überprüfung von Programmen auf Korrektheit
        \item Fehler können schnell erkannt werden
        \item Code kann direkt kommentiert werden
    \end{itemize}
    \item \textbf{Nachteile:}
    \begin{itemize}
        \item Unit Tests können unter Umständen nicht alles überprüfen
        \item Frage des Halteproblems kann auftreten
        \item erfordert Ahnung über Spezifizierung
        \item Asserts können die Performance des Programms einschränken
    \end{itemize}
\end{itemize}

Design by Contract wird nicht umfänglich eingesetzt, weil nicht viele Sprachen dieses Tool unterstützen, zudem ist die Spezifizierung ein großer Aufwand bei Projekten
die unter einer strikten Deadline stehen. Man muss auch abwiegen ob es sich lohnt, Performance einzubußen. Andere Methoden wie Unit Tests oder Test Driven Development 
haben sich stärker etabliert.

\section*{Aufgabe 2}

\begin{itemize}
    \item \textbf{Unit-Tests:}
    \item Generieren automatisch Unit tests für Funktionen und Klassen und decken dabei verschiedene Bereiche ab
    \begin{itemize}
        \item \textbf{Vorteile:}
        \begin{itemize}
            \item Automatische Testgenerierung
            \item Schnelle Fehlerentdeckung bei Entwicklungprozess
        \end{itemize}
        \item \textbf{Nachteile:}
        \begin{itemize}
            \item Mögliches fehlen von Abdeckung (Halteproblem)
        \end{itemize}
    \end{itemize}
    \item \textbf{Verifikation:}
    \item Programm wird formal verifiziert und beim Model Checking überpüft
    \begin{itemize}
        \item \textbf{Vorteile:}
        \begin{itemize}
            \item Analyse zur Designzeit
            \item Keine false Alarme
        \end{itemize}
        \item \textbf{Nachteile:}
        \begin{itemize}
            \item aufwendig
            \item nur bei programme moderater Größe anwendbar
        \end{itemize}
    \end{itemize} 
    
    \item \textbf{Dokumentation:}
    \item Generiert automatisch Dokumentation für Funktionen und Klassen in Einbezug der Bedingungen
    \begin{itemize}
        \item \textbf{Vorteile:}
        \begin{itemize}
            \item Schnelle Dokumentationserstellung
        \end{itemize}
        \item \textbf{Nachteile:}
        \begin{itemize}
            \item Unter Umständen verwirrend formuliert
        \end{itemize}
    \end{itemize}
    \item \textbf{Runtime Assertion Testing:}
    \item Generiert im Code Assertions hinein welche bei Ausführung des Codes Fehler werfen können
    \begin{itemize}
        \item \textbf{Vorteile:}
        \begin{itemize}
            \item Code wird durch asserts modifiziert
            \item Eigenes Verhalten zur Fehlerverwaltung implementierbar
        \end{itemize}
        \item \textbf{Nachteile:}
        \begin{itemize}
            \item Fehler erst bei Laufzeit bekannt
            \item teuer in Performance
        \end{itemize}
    \end{itemize}
    \item \textbf{Statische Analyse:}
    \item Übersetzt Code in SMT kompatibles Format wo es dann durch einen SMT Solver gelöst werden kann
    \begin{itemize}
        \item \textbf{Vorteile:}
        \begin{itemize}
            \item Überprüfung der Korrektheit des Programms
            \item sehr große Programme möglich
            \item Überprüfung zur Designzeit möglich
        \end{itemize}
        \item \textbf{Nachteile:}
        \begin{itemize}
            \item Mögliche False Positives und false negatives
        \end{itemize}
    \end{itemize}
\end{itemize}

\end{document}