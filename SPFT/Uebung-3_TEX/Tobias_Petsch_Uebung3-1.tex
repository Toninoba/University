\documentclass[a4paper,12pt]{article}

% Packages
\usepackage[utf8]{inputenc}
\usepackage[T1]{fontenc}
\usepackage[ngerman]{babel}
\usepackage{amsmath, amssymb}
\usepackage{graphicx}
\usepackage{geometry}
\geometry{a4paper, margin=2.5cm}

% Title and Author
\title{Übung 3.1}
\author{Tobias Petsch}
\date{}

\begin{document}

\maketitle

Maschinengestützes Beweisen basiert auf Verwendung von Computerprogrammen zur Erzeugung und Überprüfung von mathematischen Beweisen logischer Theoreme.
Hierbei wird der gesamte formale Beweis aus Schritten konstruiert.

Der Automatische Theorembeweiser soll Beweise für den Bereich der Prädikatenlogik erzeugen. Die meisten ATPs nutzen einen Beweis durch Widerspruch und beruhen auf 
einer Variante des Satzes von Herbrand. Theorembeweiser stellen Beweise für Theoreme aus Axiome und leiten über Inteferenzschritte ab.

Ein Interaktiver Theorembeweiser ist ein Beweisassistent der aus Hinweisen vom Menschen Beweise überprüft. Ein formaler Beweis wird überprüft ob er ein
gegebenes Theorem beweist. Hier fällt auch KIV rein, da es nur mithilfe von Gesetzen und Hinweistellungen vom Menschen ein formalen Beweis überprüft.

\end{document}