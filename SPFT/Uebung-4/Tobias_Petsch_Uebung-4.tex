\documentclass[a4paper,12pt]{article}

% Packages
\usepackage[utf8]{inputenc}
\usepackage[T1]{fontenc}
\usepackage[ngerman]{babel}
\usepackage{amsmath, amssymb}
\usepackage{graphicx}
\usepackage{geometry}
\geometry{a4paper, left=25mm, right=25mm, top=25mm, bottom=25mm}

% Title and Author
\title{Übung 4}
\author{Tobias Petsch}
\date{}

\begin{document}

\maketitle

\section*{Aufgabe 1}
\subsection*{a)}

\[
\forall x (F_r(x) \lor F_s(x))
\]

\[
\forall x \forall y (F_r(x) \land F_r(y) \land K(x,y)) \lor (F_s(x) \land F_s(y) \land K(x,y)) \rightarrow \bot 
\]

\subsection*{b)}
\[
A \land \forall x \forall y \left( (F_r(x) \land F_s(y)) \rightarrow K(x, y) \right)
\]

$A$ = Ergbenis aus Teilaufgabe a).


% Inhalt der Aufgabe 1

\section*{Aufgabe 2}
% Inhalt der Aufgabe 2



$z$ ist ggT von $x$ und $y$, wenn:

1. $z$ teilt $x$ und $y$:  
\[
\exists a\, (x = z \cdot a) \land \exists b\, (y = z \cdot b)
\]

2. Jeder gemeinsame Teiler $d$ von $x$ und $y$ teilt auch $z$:  
\[
\forall d \left( (\exists a\, (x = d \cdot a) \land \exists b\, (y = d \cdot b)) \rightarrow \exists c\, (z = d \cdot c) \right)
\]

Formel:
\[
A(x, y, z) := 
\left( \exists a\, (x = z \cdot a) \land \exists b\, (y = z \cdot b) \right) 
\land 
\left( \forall d \left( (\exists a\, (x = d \cdot a) \land \exists b\, (y = d \cdot b)) \rightarrow \exists c\, (z = d \cdot c) \right) \right)
\]


Die Formel stellt sicher, dass $z$ sowohl $x$ als auch $y$ teilt, also ein gemeinsamer Teiler ist. 
Zusätzlich wird gefordert, dass jeder andere gemeinsame Teiler $d$ auch $z$ teilt. 
Damit ist $z$ der größte gemeinsame Teiler.

\section*{Aufgabe 3}

\subsection*{a)}
\[
\forall x \forall y \left( \text{Kind}(y, x) \leftrightarrow \text{Elternteil}(x, y) \right)
\]

\subsection*{b)}
\[
\forall x \forall y \left( \text{Verheiratet}(x, y) \rightarrow \neg \text{Geschwister}(x, y) \right)
\]

\subsection*{c)}
\[
\forall x \forall z \left( \text{Großelternteil}(x, z) \leftrightarrow \exists y \left( \text{Elternteil}(x, y) \land \text{Elternteil}(y, z) \right) \right)
\]

\section*{Aufgabe 4}

\subsection*{a)}
\[
\forall x \forall y \left( \text{Geschwister}(x, y) \rightarrow \text{Geschwister}(y, x) \right)
\]

\subsection*{b)}
\[
\forall x \left( \neg \text{Geschwister}(x, x) \right)
\]



\end{document}