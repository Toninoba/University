\documentclass[a4paper,12pt]{article}

% Packages
\usepackage[utf8]{inputenc}
\usepackage[T1]{fontenc}
\usepackage[ngerman]{babel}
\usepackage{amsmath, amssymb}
\usepackage{graphicx}
\usepackage{geometry}
\geometry{a4paper, left=25mm, right=25mm, top=25mm, bottom=25mm}

% Title and Author
\title{Übung 2.6}
\author{Tobias Petsch}
\date{}

\begin{document}

\maketitle

Wir erstellen Variablen
\begin{description}
    \item[$S$: ] Steuererhoehungen
    \item[$P$: ] Preisinstabilitaet
    \item[$H$: ] Staatshaushalt kuerzen
\end{description}


Wir erstellen Aussagen

aus (i) folgt $S \Rightarrow P$ \\
aus (ii) folgt $\neg S \Rightarrow H$ \\ 
aus (iii) folgt $\neg H \land P \Rightarrow S$ \\

Wir beweisen durch Widerspruch ob (c) eine logische Folgerung ist. \\
Wir nehmen an $\neg S$ \\
Daraus folgt die Kontraposition $\neg S \Rightarrow \neg P$ \\
In Aussage (iii) entsteht so $(\neg H \land \neg P)$ was $false$ ist wodurch die Implikation wahr wird. \\
Aussage (ii) bleibt unklar und somit kann nicht entschieden werden ob Steuern erhoeht werden muessen, da beide Ergebnisse korrekt sind. \\
Daraus folgt das Ohnemoosnixlos Gedanke falsch ist. 

% Hier kommt der Inhalt der Aufgabe hin.

\end{document}