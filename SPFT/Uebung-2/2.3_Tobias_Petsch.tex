\documentclass[a4paper,12pt]{article}

\usepackage[utf8]{inputenc}
\usepackage[T1]{fontenc}
\usepackage[ngerman]{babel}
\usepackage{amsmath, amssymb}
\usepackage{graphicx}
\usepackage{geometry}
\geometry{a4paper, margin=2.5cm}

\title{Aufgabe 2.3}
\author{Tobias Petsch}
\date{}

\begin{document}

\maketitle

\section*{(a)}

\[
TRUE \models (A \land B) \Rightarrow ((A \lor \neg B) \land B)
\]

Wir sollen zeigen, dass $(A \land B) \Rightarrow ((A \lor \neg B) \land B)$ immer wahr ist.
Dazu formen wir die Formel in ein Sequent um. \\
$\vdash (A \land B) \Rightarrow ((A \lor \neg B) \land B)$

Da Implikationen im Sequenzenkalkuel als Regeln behandelt werden entsteht \\
$A \land B \vdash (A \lor \neg B)$ \\

$A \land B$ kann als Annahme $A,B$ zusammengefasst werden, wir erhalten nun die Sequenten 
$A,B \vdash A \lor \neg B \text{ und } A,B \vdash B$ \\
Da $A \vdash A$ trivial ist, ist der erste Teil abgehakt, da True oder etwas anderes immer true ergibt.
Die Rechte Seite ist ebenfalls trivial da als Praemisse B auf der linken Seite steht. Da beide Teilsequenten beweisbar
sind ist die gesamte Formel gueltig und erfuellbar ist. 


\section*{(b)}

\[
A \land (B \Rightarrow \neg A) \land (A \Rightarrow B) \models FALSE
\]

da hier durch Umformung ein Sequent auf rechter Seite folgt und die rechte Seite leer bleibt,
heisst das es soll ein Widerspruch folgen. Alle Konjuntionen werden nun zu Annahmen. Aus den Implikationen
werden Disjunktionen \\
$B \Rightarrow \neg A \rightarrow \neg B \lor \neg A $ \\
$A \Rightarrow B \rightarrow \neg A \lor \ B$ \\
Aus diesen Annahmen entsteht eine widerspruechliche Menge, da aus $A \Rightarrow B$ und $B \Rightarrow \neg A$ 
es nun heissen wuerde, dass $A \Rightarrow \neg A$ folge. Dies fuehrt aber zu einem Widerspruch, weswegen die Formel Unerfuellbar und nicht gueltig wird. 


\end{document}