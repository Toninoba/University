\documentclass[a4paper,12pt]{article}

\usepackage[utf8]{inputenc}
\usepackage[T1]{fontenc}
\usepackage[ngerman]{babel}
\usepackage{amsmath, amssymb}
\usepackage{graphicx}
\usepackage{geometry}
\geometry{a4paper, margin=2.5cm}

\title{Aufgabe 2.3}
\author{Tobias Petsch}
\date{}

\begin{document}

\maketitle

\section*{(a)}

\[
TRUE \models (A \land B) \Rightarrow ((A \lor \neg B) \land B)
\]

Wir sollen zeigen, dass $(A \land B) \Rightarrow ((A \lor \neg B) \land B)$ immer wahr ist.
Dazu formen wir die Formel in ein Sequent um. \\
$\vdash (A \land B) \Rightarrow ((A \lor \neg B) \land B)$

Da Implikationen im Sequenzenkalkuel als Regeln behandelt werden entsteht \\
$A \land B \vdash (A \lor \neg B)$ \\

$A \land B$ kann als Annahme $A,B$ zusammengefasst werden, wir erhalten nun die Sequenten 
$A,B \vdash A \lor \neg B \text{ und } A,B \vdash B$ \\
Da $A \vdash A$ trivial ist, ist der erste Teil abgehakt, da True oder etwas anderes immer true ergibt.
Die Rechte Seite ist ebenfalls trivial da als Praemisse B auf der linken Seite steht. Da beide Teilsequenten beweisbar
sind ist die gesamte Formel gueltig und erfuellbar ist. 

\subsection*{Lösung}

Beim Sequenzenkalkül formen wir von unten nach oben und führen von Axiomen zur Formel, falls sie
auf Axiome reduzierbar ist dann ist sie auch erfüllbar.

Man beginnt mit der gesamten Formel als untersten Teil.
Dann werden einzelne Regeln aus einer Übersicht angewendet.
Scheinbar hab ich das ähnlich gemacht nur die Formalien sind falsch.
WICHTIG, Anstelle $\models$ $\vdash$ nutzen beim beweisen. 
Formeln muss man kennen also auswendig lernen.

Lösung erhällt man wenn entweder gleiche Teile auf beiden Seiten stehen, oder false left oder true right irgendwo steht.
Bsp. für Axiom $B \vdash B$, es reicht wenn ein Symbol auf beiden Seiten steht, getrennt von anderen mit einem Komma.

Wenn es herleitbar ist dann ist es gültig?
Wenn $\phi \models FALSE$ herleitbar ist dann ist so unerfüllbar war alle Belegungen false ergeben
\section*{(b)}

\[
A \land (B \Rightarrow \neg A) \land (A \Rightarrow B) \models FALSE
\]

da hier durch Umformung ein Sequent auf rechter Seite folgt und die rechte Seite leer bleibt,
heisst das es soll ein Widerspruch folgen. Alle Konjuntionen werden nun zu Annahmen. Aus den Implikationen
werden Disjunktionen \\
$B \Rightarrow \neg A \rightarrow \neg B \lor \neg A $ \\
$A \Rightarrow B \rightarrow \neg A \lor \ B$ \\
Aus diesen Annahmen entsteht eine widerspruechliche Menge, da aus $A \Rightarrow B$ und $B \Rightarrow \neg A$ 
es nun heissen wuerde, dass $A \Rightarrow \neg A$ folge. Dies fuehrt aber zu einem Widerspruch, weswegen die Formel Unerfuellbar und nicht gueltig wird. 


\end{document}