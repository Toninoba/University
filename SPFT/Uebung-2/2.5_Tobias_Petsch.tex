\documentclass[a4paper,12pt]{article}

\usepackage[utf8]{inputenc}
\usepackage[T1]{fontenc}
\usepackage[ngerman]{babel}
\usepackage{amsmath, amssymb}
\usepackage{graphicx}
\usepackage{geometry}
\geometry{a4paper, margin=2.5cm}

\title{Übung 2.5}
\author{Tobias Petsch}
\date{}

\begin{document}

\maketitle

Wir definieren Atomare Aussagen \\
\begin{description}
    \item[$L$:] Der linke Weg führt nach Delphi
    \item[$M$:] Der mittlere Weg führt nach Delphi
    \item[$R$:] Der rechte Weg führt nach Delphi
    \item[$V$:] Der rechte Weg führt ins Verderben
    \item[$\neg M$:] Der mittlere Weg fuehrt nicht nach Delphi 
\end{description}

\textit{„Wisse, wenn der mittlere Weg in die Irre führt, so gelangst du auf dem linken und dem rechten Wege nach Delphi.“}

Das bedeutet: $\neg M \rightarrow (L \land R)$

\textit{„Der Weise erkennt, sollte vom linken und mittleren Wege höchstens einer nach Delphi führen, so führt der Rechte sicher in die Leere.“}

Das bedeutet: $\neg (L \land M) \rightarrow \neg R$

\textit{„Wenn der Suchende jedoch auf dem rechten oder linken Weg nach Delphi gelangt, dann führt der mittlere mit Sicherheit ins Verderben.“}

Das bedeuet: $(R \lor L) \rightarrow V$

% Hier beginnt der Inhalt der Aufgabe.

\end{document}