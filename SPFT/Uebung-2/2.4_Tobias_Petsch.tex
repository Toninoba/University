\documentclass[a4paper,12pt]{article}

% Packages
\usepackage[utf8]{inputenc}
\usepackage[T1]{fontenc}
\usepackage[ngerman]{babel}
\usepackage{amsmath, amssymb}
\usepackage{graphicx}
\usepackage{geometry}
\usepackage{hyperref}

% Page layout
\geometry{a4paper, margin=2.5cm}

% Title
\title{Übung 2.4}
\author{Tobias Petsch}
\date{}

\begin{document}

\maketitle

\[
\neg A \models (B \Rightarrow \neg A) \land (A \Rightarrow B)
\]

Wir sollen beweisen dass $\neg A$ eine Loesung der Formel ist, nun koennen wir diese Annehme umformen zu einem Sequent

\[
\neg A \vdash (B \Rightarrow \neg A) \land (A \Rightarrow B)
\]

Das logische und kann ersetzt werden, sodass wir aus $\neg A$ nun $(B \Rightarrow \neg A)$ und $(A \Rightarrow B)$
zeigen muessen. Das erste Teilsequent kann umgeformt werden \\
\[
\neg A \vdash B \Rightarrow \neg A \rightarrow \neg A,B \vdash \neg A
\] 
Da aus $\neg A \rightarrow \neg A$ folgt, ist der erste Teil schonmal erfuellbar, das zweite Teilsequent wird gleichermasen umgeformt.

\[
\neg A \vdash A \Rightarrow B \rightarrow \neg A,A \vdash B
\]

Hier stossen wir auf einen Widerspruch der Annahmen, wodurch wir alles annehmen koennen, weswegen auch der zweite Teil erfuellbar ist.
Dadurch ist die gesamte Formel erfuellbar

% Hier kommt der Inhalt der Aufgabe hin.

\end{document}