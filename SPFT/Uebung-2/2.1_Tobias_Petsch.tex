\documentclass[a4paper,12pt]{article}
\author{Tobias Petsch}

\usepackage[utf8]{inputenc}
\usepackage[T1]{fontenc}
\usepackage[ngerman]{babel}
\usepackage{amsmath, amssymb}
\usepackage{graphicx}
\usepackage{geometry}

\geometry{a4paper, margin=1in}
\setlength{\parindent}{0pt}

\title{Aufgabe 2.1}
\date{}
\begin{document}
\maketitle


\section*{a)}
\[
(A \lor \neg B) \land B
\]

Mit $B$ als Einheitsklausel folgt das $B$ true sein muss, da wir aufgrund der KNF die Literale
voneinander trennen. Einsetzen in $A \lor \neg B$ ergibt $A \lor false$, was wiederrum $A$ ergibt.
Da $A$ wieder eine Einheitsklausel ist und so auch erfüllbar ist, kann die gesamte Formel erfüllt werden. \\

Aufwand: Da beide Operationen Unit-Propagation waren entsprach der Aufwand O(1) für jede propagation und insgesamt
O(n) da kein Backtracking betrieben werden musste.


\subsection*{Lösung}
DP - Algorithmus \\
Belege wie eine Art Baumstruktur. Belege A mit wahr $h(A)=T$ und $h(A)=F$ und gehe wie einen Binärbaum runter und schaue was am Ende übrig bleibt.
Dann belege $B$ mit True und False und schaue wie die Blätterknoten aussehen. Daraus folgt eine Variablenbelegung $\{1,1\}$ heißt $A=T$ und $B=T$.
Erfüllbar ist eine Formel dann, wenn mindestens eine Belegung sie wahr machen würde. Die Lösungsmenge darf also nicht Leer sein.


\section*{b)}
\[
A \lor \neg (B \land \neg C) \Leftrightarrow C \Rightarrow A
\]

Nehme an $A = true$ dann folgt daraus $true \lor ... \Leftrightarrow ...$ und
auf der rechten Seite $C \Rightarrow true$, da dies ebenfalls zu $true$ gekürzt werden kann 
bleibt $true \Leftrightarrow true$ übrig. Daraus folgt das die Formel erfüllbar ist.

Aufwand: bleibt bei O(n) da kein Backtracking nötig war.

\section*{c)}
\[
(A \lor B) \land (\neg A \lor B) \land (A \lor \neg B)
\]

Aus der KNF folgen drei einzelne Formeln 
\begin{align}
    K1 &= (A \lor B) \\
    K2 &= (\neg A \lor B) \\
    K3 &= (A \lor \neg B)
\end{align}

Wir wählen zufällig A aus und setzen es auf true. Daraus entstehen folgende Formeln

\begin{align}
    K1 &= (true \lor B) \rightarrow \text{erfüllt}\\
    K2 &= (false \lor B) \rightarrow \text{es bleibt B übrig}\\
    K3 &= (true \lor \neg B) \rightarrow \text{erfüllt}
\end{align}

Nun bleibt B als Einheitsklausel übrig und wir können Unit-Propagation ausführen.
Daraus folgt das auch K2 erfüllbar wird und so alle Formeln erfüllbar sind.

Aufwand: Kein Backtracking nötig, daraus folgt ein Aufwand von O(n)

\end{document}