\documentclass[a4paper,12pt]{article}

\usepackage[utf8]{inputenc}
\usepackage[T1]{fontenc}
\usepackage[ngerman]{babel}
\usepackage{amsmath, amssymb}
\usepackage{graphicx}
\usepackage{geometry}

\geometry{a4paper, margin=1in}

\title{Übung 1 - SPFT}
\author{Tobi}
\date{\today}

\begin{document}

\maketitle

\section*{Aufgabe 1}
Software muss spezifiziert werden, damit diese korrekt funktioniert. Viele Systeme bedarfen genaue Sicherheitssysteme
welche für die Sicherheit vieler Menschen wichtig ist. Außerdem kann ``falsche'' Software zum Verlust von Geld und Geschwindigkeit in
der Industrie führen.

\section*{Aufgabe 2}
informale Spezifikation: hat den Vorteil das diese von vielen Leuten verstanden wird, da es eine natürlich gesprochene Sprache ist, Nachteilig
ist hier das die Sprache auch mehrdeutig sein kann.
formale Spezifikation: hat den Vorteil das sie sehr genau ist und man viele Systeme sehr genau spezifizieren kann, dafür funktioniert diese Art 
nicht für alle Systeme und sie kann auch nicht direkt von jedem verstanden werden.

\section*{Aufgabe 3}
Die Spezifikation beschreibt was die Software tun soll, während die Verifikation prüft ob die Software tatsächlich das tut was sie soll.
Bsp. Login System, Die Spezifikation ist hier das ein Benutzer sich mit Email und Passwort anmelden können, das ein Passwort mindestens 8 
Zeichen lang sein muss und das nach 3 Fehlversuchen der Account gesperrt wird. Die Verifikation testet genau diese Vorgaben ob sie tatsächlich
auch so eintreten.

\section*{Aufgabe 4}
Die syntaktische Korrektheit beschreibt ob etwas richtig geschrieben ist, aber es wird nicht kontrolliert ob es auch korrekt ist.
Die semantische Korrektheit überprüft dann ob das richtig geschrieben auch korrekt und sinnvoll ist.\\
Bsp. C-Code, deutscher Satz

\section*{Aufgabe 5}
Fobos-Grunt (2011):\\
Ursache: Softwarefehler im Steuerungssystem führte zu einem Reset der Bordrechner\\
Folge: Die Raumsonde konnte ihre Mission nicht beginnen und stürzte ab, Verlust von ca. 120 Mio US Dollar\\
Vermeidbar: Ja, eine präzise Spezifikation und Verifikation kritischer Softwarepfade hätte helfen können, Fehlfunktionen beim Systemstart zu vermeiden\\

Galileo Debakel (2014):\\
Ursache: Softwarefehler in der Steuerung der Trägerrakete Sojus-Fregat, der zu falscher Bahnführung der Galileo-Satelliten führte.\\
Folge: Zwei teure Navigationssatelliten wurden in eine nutzlose Umlaufbahn gebracht.\\
Vermeidbarkeit durch Spezifikation: Ja, eine fehlerfreie Spezifikation und bessere Systemintegrationstests hätten den Steuerungsfehler vermutlich verhindern können.\\

Germanwings Flug 9525 (2015)\\
Ursache: Der Co-Pilot brachte das Flugzeug absichtlich zum Absturz. Die Software ließ ein manuelles Absenken der Flughöhe ohne Rückfrage zu.\\
Folge: Absturz mit 150 Toten.\\
Vermeidbarkeit durch Spezifikation: Möglicherweise, eine sicherheitsspezifische Spezifikation hätte fordern können, dass bei drastischen Höhenänderungen zusätzliche Prüfmechanismen greifen (z.~B. Zweitautorisierung).

\section*{Aufgabe 6}

\begin{table}[h!]
    \centering
    \begin{tabular}{|l|l|l|}
    \hline
    & Aussagenlogischer Ausdruck & Tautologie oder Kontradiktion \\ \hline
    a) & ja & weder noch \\ \hline
    b) & ja & Tautologie \\ \hline
    c) & ja & weder noch \\ \hline
    d) & ja & weder noch \\ \hline
    e) & ja & Kontradiktion \\ \hline
    f) & nein & geht nicht \\ \hline
    g) & ja & weder noch \\ \hline
    h) & ja & Tautologie \\ \hline
    \end{tabular}
    \end{table}

\section*{Aufgabe 7}
    \subsection*{a)}
    
    \centering
    \begin{tabular}{|l|l|l|l|}
    \hline
    A & B & C & $(A \implies B) \land (B \implies \neg A)$ \\ \hline
    f & f & f & w \\ \hline
    f & f & t & w \\ \hline
    f & t & f & f \\ \hline
    f & t & t & f \\ \hline
    t & f & f & f \\ \hline
    t & f & t & f \\ \hline
    t & t & f & f \\ \hline
    t & t & t & f \\ \hline
    \end{tabular}
    
    \subsection*{b)}
    
    \centering
    \begin{tabular}{|l|l|l|l|}
    \hline
    A & B & C & $A \implies B \implies C$ \\ \hline
    f & f & f & w \\ \hline
    f & f & t & w \\ \hline
    f & t & f & w \\ \hline
    f & t & t & w \\ \hline
    t & f & f & w \\ \hline
    t & f & t & w \\ \hline
    t & t & f & f \\ \hline
    t & t & t & w \\ \hline
    \end{tabular}
    
    \subsection*{c)}
    
    \centering
    \begin{tabular}{|l|l|l|l|}
    \hline
    A & B & C & $A \land (B \implies \neg A) \land (A \implies B)$ \\ \hline
    f & f & f & f \\ \hline
    f & f & t & f \\ \hline
    f & t & f & f \\ \hline
    f & t & t & f \\ \hline
    t & f & f & f \\ \hline
    t & f & t & f \\ \hline
    t & t & f & f \\ \hline
    t & t & t & f \\ \hline
    \end{tabular}
    
    \subsection*{d)}
    
    \centering
    \begin{tabular}{|l|l|l|l|}
    \hline
    A & B & C & $A \iff B \iff C$ \\ \hline
    f & f & f & w \\ \hline
    f & f & t & f \\ \hline
    f & t & f & f \\ \hline
    f & t & t & w \\ \hline
    t & f & f & f \\ \hline
    t & f & t & w \\ \hline
    t & t & f & w \\ \hline
    t & t & t & f \\ \hline
    \end{tabular}
    


\end{document}