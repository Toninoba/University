\documentclass[a4paper,12pt]{article}

\usepackage[utf8]{inputenc}
\usepackage[T1]{fontenc}
\usepackage[ngerman]{babel}
\usepackage{amsmath, amssymb}
\usepackage{stmaryrd}
\usepackage{graphicx}
\usepackage{geometry}
\geometry{left=2.5cm, right=2.5cm, top=2.5cm, bottom=2.5cm}

\title{Übung 5}
\author{Tobias Petsch}
\date{}

\begin{document}

\maketitle

\section*{Aufgabe 3}

Ein Programm ist korrekt, wenn es für jede erlaubte Eingabe die laut Spezifikation richtige Ausgabe liefert = Semantik. 
Man unterscheidet dabei zwischen partieller und totaler Korrektheit: 
Partielle Korrektheit bedeutet, dass das Programm, wenn es terminiert, die korrekte Ausgabe liefert. 
Totale Korrektheit verlangt zusätzlich, dass das Programm bei jeder gültigen Eingabe auch wirklich terminiert. Nur dann ist das Programm vollständig korrekt.



\section*{Aufgabe 6}

\subsection*{(a)}

Gegeben:
\[
\{x \geq 5\}~x := x \cdot 2~\{x \geq 8\}
\]

Nach der \textbf{Zuweisungsregel} gilt:
\[
\{x \cdot 2 \geq 8\}~x := x \cdot 2~\{x \geq 8\}
\Rightarrow x \geq 4
\]

Da:
\[
x \geq 5 \Rightarrow x \geq 4
\]

gilt die \textbf{Konsequenzregel}, somit:
\[
\{x \geq 5\}~x := x \cdot 2~\{x \geq 8\}
\quad 
\]

\subsection*{(b)}

\[
\left\{
  2 \leq n \leq 3 \land n \in \mathbb{N}
\right\}
~\texttt{if } n = 2 \texttt{ then } m := 4 \texttt{ else } m := 9~
\left\{
  m = n^2
\right\}
\]

\textbf{Fall 1:} $n = 2$ \\
Dann $m := 4$, und $n^2 = 4$ $\Rightarrow m = n^2$ 

\textbf{Fall 2:} $n = 3$ \\
Dann $m := 9$, und $n^2 = 9$ $\Rightarrow m = n^2$ 

\textbf{Mit der If-Regel:}
\[
\{2 \leq n \leq 3 \land n \in \mathbb{N}\}~
\texttt{if } n = 2 \texttt{ then } m := 4 \texttt{ else } m := 9~
\{m = n^2\}
\quad 
\]

\subsection*{(c)}

\[
\{x = x_0 \land y = y_0\}~
t := y;\; y := x;\; x := t~
\{x = y_0 \land y = x_0\}
\]

\textbf{Rückwärtsanalyse:}

\begin{itemize}
  \item $x := t$ benötigt $t = y_0$
  \item $y := x$ benötigt $x = x_0$
  \item $t := y$ benötigt $y = y_0$
\end{itemize}

Eingangsbedingung ist also:
\[
x = x_0 \land y = y_0
\Rightarrow \text{Tripel korrekt} \quad
\]

\subsection*{(d)}

\[
\{x > n \land x < n\}~x := x + 1~\{x < n\}
\]

Da:
\[
x > n \land x < n \equiv \text{false}
\Rightarrow \text{Vorbedingung ist unerfüllbar}
\]

Nach der \textbf{Konsequenzregel} gilt jedes Tripel mit \texttt{false} als Vorbedingung:
\[
\{\text{false}\}~x := x + 1~\{x < n\}
\quad 
\]

\end{document}